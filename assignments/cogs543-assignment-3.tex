\documentclass[10pt,a4paper]{exam}

\usepackage{umut}
\usepackage{uling}
\usepackage{mathptmx}
\usepackage{uprog}
\usepackage{uhref}
\usepackage{usynsem}

%\printanswers

\pagestyle{headandfoot}
	\lhead{Cogs 543 \\ Computational Semantics}
	\chead{Assignment 3}
	\rhead{Fall 2023\\ due Dec 12, before class}
\lfoot{}
\pointname{\%}

\begin{document}
\qformat{\bf Q \thequestion.%
\ifthenelse{\equal{\thepoints}{}}{}{\quad (\thepoints)} \hfill}

% \makebox[\textwidth]{Name of the Student:\enspace\hrulefill}

\vspace{10pt}

% \begin{center}
% \fbox{\parbox{6in}{\bf\centering 4 questions in 150 minutes}}
% \end{center}
%

\vspace{90pt}


Here is your lexicon:


\begin{align*}
	\text{walks}   &:=& \sysm{\lambda x.walk'x}                          &::& \sysm{et}\\
	\text{loves}   &:=& \sysm{\lambda x\lambda y.love'xy}                &::& \sysm{e(et)}\\
	\text{reads}   &:=& \sysm{\lambda x\lambda y.read'xy}                &::& \sysm{e(et)}\\
	\text{John}    &:=& \sysm{j'}                                        &::& \sysm{e}\\
	\text{Mary}    &:=& \sysm{m'}                                        &::& \sysm{e}\\
	\text{woman}   &:=& \sysm{\lambda x.woman'x}                         &::& \sysm{et}\\
	\text{book}    &:=& \sysm{\lambda x.book'x}                          &::& \sysm{et}\\
	\text{blue}    &:=& \sysm{\lambda p\lambda x.blue'x\land p\cnct{}x}  &::& \sysm{et(et)}\\
	\text{is}      &:=& \sysm{?}                                         &::& \sysm{?}\\
	\text{no}      &:=& \sysm{\lambda p\lambda q.\neg(\exists x. px \land qx)} &::& \sysm{et(ett)}\\
	\text{a}       &:=& \sysm{\lambda p\lambda q.\exists x. px \land qx} &::& \sysm{et(ett)}\\
	\text{every}   &:=& \sysm{\lambda p\lambda q.\forall x. px \cond qx} &::& \sysm{et(ett)}\\
	\text{QOBJ}    &:=_{LEX}& \sysm{?} &::& \sysm{?}
\end{align*}

\begin{questions}
\question

The lexicon has only subject position interpretations for the
quantiifers \emph{a}, \emph{every}, and \emph{no}. You are required to
write a lexical rule as a lambda term (see \sysm{QOBJ}) that applies
to quantifier interpretations to turn them to object position
interpretations. With this in place, you should be able to derive the
meaning of sentences like:

\ex. (John (reads (every book)))

\question

Derive the meaning of:

\ex. ((No woman) (reads (every (blue book)))).


\question

The interpretation for the adjective \emph{blue} is suitable for its
attributive use only. By this lexicon, we assume that this is its
basic interpretation. Propose an interpretation for the copula
\emph{is}, so that you can derive the meaning of sentences like the
following:

\ex. ((No book) (is blue)).



\end{questions}


Please submit all your answers in Lambda Calculator.

\end{document}
